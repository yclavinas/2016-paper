The GAModel is a method which aims to generate forecasts by using only
Evolutionary Computation (EC). However, the GAModel was limited by the
very high number of parameters, and the subsequent large search
space. This project main goal is to refine the GAModel ideas,
objective to have a greater ability to predict the future behavior of
groups of earthquakes events by overcoming this limitations.

This document sumamarizes the GAModel and proposes three new method
based on the GAModel. The first method is called the
ReducedGAModel. By this, we expecte to reduce the impact of the amount
of earthquakes parameters and the size of the search space generating
a forecast model faster. For that we used a different genome
representation than the representation used in the GAModel. In the
GAModel, the genome is a real valued genome, with every gene
coresponds to a specific bin in the model. In the ReducedGAModel, the
genome is a list of earthquake events locatation. Each element of the
genome has a correspondent bin in the model.

We also wanted to do a hybridization, a association of EC and
geophysical knowledge, of the GAModel and the ReducedGAModel with
empirical laws, such as the modified Omori-Utsu formula. These new
methods names are Emp-GAModel and Emp-ReducedGAModel. This
hybridization is performed in two phases. The first one, is to obtain
forecast models obtained by the GAModel and the ReducedGAModel. After
that, in the second phase, these models are all refined by the same
group of geophysical formulas. We expect that we will be able to
generate not only faster model but more accurated ones, because we
will reduce the search space and increase its learning rate with the
formulas.


The models generated by these four method were evaluated and compared
based on the predictabily experiments framework proposed by the
"Collaboratory for the Study of Earthquake Predictability" (CSEP), an
international effort to standardize the study and testing of
earthquake forecasting models. The experiments were designed to
compare 1-year earthquake rate forecasts for four regions in Japan in
using the data from the Japan Meteorological Agency (JMA) earthquake
catalog.